% This is LLNCS.DEM the demonstration file of
% the LaTeX macro package from Springer-Verlag
% for Lecture Notes in Computer Science,
% version 2.4 for LaTeX2e as of 16. April 2010
%
\documentclass{llncs}
%
\usepackage{makeidx}  % allows for indexgeneration
\usepackage{llncsdoc}
\usepackage{algorithmic}
\usepackage{amssymb}
\usepackage{graphicx}
\usepackage{wrapfig}
\usepackage{cite}
\usepackage{amsmath}
\usepackage[usenames,dvipsnames]{xcolor}
\usepackage[parfill]{parskip} 
\usepackage{float}
\usepackage{caption}
%\restylefloat{table}

\newcommand{\Mod}[1]{\ (\text{mod}\ #1)}
\newcommand{\todo}[1]{{\color{red}{TODO #1}}}
\captionsetup[table]{name=Tabel}
%\restylefloat{table}
%
\begin{document}
\pagestyle{empty}
%\centerline{\bf UNGIVERSITY OF BUCHAREST}

%\centerline{\bf FACULTY OF MATHEMATICS AND COMPUTER SCIENCE}

%
%
\title{Data Structures and Algorithms}
%
\titlerunning{Data Structures and Algorithms}  % abbreviated title (for running head)
%                                     also used for the TOC unless
%                                     \toctitle is used
%
\author{Drago\c{s} Alin Rotaru}
%TODO - nume final
\authorrunning{and D.A.Rotaru} % abbreviated author list (for running head)
%
%%%% list of authors for the TOC (use if author list has to be modified)
%\tocauthor{}
%
\institute{Computer Science, University of Bucharest, Romania\\
}


\maketitle              % typeset the title of the contribution


\newpage
%----------------------------------------------------------------
%----------------------------------------------------------------
%----------------------------------------------------------------
%----------------------------------------------------------------
%----------------------------------------------------------------

\section{Introduction}
\label{sec:intro}

These seminar notes contain my overview of the Data Structures and Algorithms course held at University of Bucharest. Because the course is based on heavy theoretic lectures, I tried a more practical approach to present some of the notions by discussing problems which arise natural from the main course.

Most of the problems come from a romanian website specialized on programming contests as well as codeforces or topcoder \cite{website:infoarena, website:topcoder, website:codeforces}. Of course, there are more interesting problems to tackle, but unfortunately I limit to the course material although sometimes I will talk about some ad-hoc problems.


\section{Seminar I}
\label{sec:seminar1}

Synthesise first 2 courses:
\begin{itemize}
	\item Basic notions of time and memory complexity.
	\item Stacks and Queues.
\end{itemize}

\subsection{Sketch}
	What is an algorithm? How can we measure the time complexity of a program? Examples (Choosing every pair of elements and erathosthene sieve).
	Introduction to stacks and queues. Details about their implementation and a short tutorial in STL. Can also talk about circular queues and double ended queues.

\subsection{Partial Sums without subtracting}

\subsection{Checking if an expression has brackets in right order}

\subsection{Emulate a queue using 2 stacks}

\subsection{Editor \cite{website:infoarena/editor}}

\subsection{Alee \cite{website:infoarena/alee}}

\subsection{Trompeta \cite{website:infoarena/trompeta}}
Given N digits, find the maximum number with K digits such that the digits preserve the initial order.

\subsection{Tsunami \cite{website:infoarena/tsunami}}
Given a matrix NxM with digits less than a value $K (<= 10)$. Find the number of cells with value less than a threshold $T (T <= K)$ such that when you start from 0 you always go up to digits $<= T$.

\subsection{Add or Multiply}

Take a number X. You can add, subtract, multiply it by numbers in a set $S$. Find the shortest number of steps required to reach from $X$ to $Y$.

\subsection{Take-Out \cite{website:edu-pl:take-out}}

N blocks, each black or white. You can remove at a time $k$ white blocks and $1$ black block such that there is no gap between the removed blocks.
\subsection{Devices}
	You are given a row of n devices, each consuming some subset of k<=8 different resources when turned on, and producing some amount of energy when turned on. For each l from 1 to n you need to find the smallest r such that it's possible to turn on some devices from the segment [l;r] such that no two devices turned on consume the same resource, and that the total energy of the devices turned on is at least z \cite{website:petr1}.

\section{Seminar II}
\label{sec:seminar2}

\begin{itemize}
	\item Divide and conquer, merge-sort, estimating complexity
\end{itemize}

\subsection{Inv \cite{website:infoarena/inv}}
\subsection{Stergeri \cite{website:infoarena/stergeri}}
\subsection{Muzeu \cite{website:infoarena/muzeu}}
\subsection{Binar \cite{website:infoarena/binar}}


\section{Seminar III}

Binary search. Fast exponentiation. K'th element. Time complexity proofs (at least sketches).

\subsection{Classic Task \cite{website:infoarena/classictask}}
\subsection{Nrtri \cite{website:infoarena/nrtri}}
\subsection{Sdo \cite{website:infoarena/sdo}}
	As a further challenge, consider that you can't store $O(N)$ elements. Can you do dynamic medians?
\subsection{Loto \cite{website:infoarena/loto}}

\section{Seminar IV}

Heaps. Binary Heaps. Fibonacci Heaps.

Basic heap operations (Insert, Extract-Min, Merge).

Time, space complexity analysis.
Heap reconstruction in $O(N)$.

\subsection{Sdo \cite{website:infoarena/sdo}}
Can you find the median using heaps? Can you do this dinamically?

\subsection{Merge $K$ lists \cite{website:leetcode/merge}}
You have $K$ sorted arrays. Can you merge them quickly?

\subsection{Sort using heaps}
Classic sort algorithm, this time using heaps. Almost sorted list, can you sort it more efficiently than classic?

%things to cover: divide and conquer, recurrence relations, binary search trees (AVL), huffman encoding, quick sort, heapsort, hashing, graphs, mst
%some problems as backups:
%telina: have a string s(1)...s(n): choose for each i (n->1) if you keep the string or reverse it's suffix
%odd number of times in logn; elements come in pairs;
%
% ---- Bibliography ----
%
%\begin{thebibliography}{5}
%

\section{Back-up problems}

\subsection{Secv6 \cite{website:infoarena/secv6}}
\subsection{Dynamic GCD \cite{website:timus/gcd}}
\subsection{Matrice 2 \cite{website:infoarena/matrice2}}
\subsection{Cabane \cite{website:infoarena/cabane}}
\subsection{Minim2 \cite{website:infoarena/minim2}}


\clearpage
\bibliographystyle{splncs}
\bibliography{llncs}
%\end{thebibliography}

\end{document}
