% This is LLNCS.DEM the demonstration file of
% the LaTeX macro package from Springer-Verlag
% for Lecture Notes in Computer Science,
% version 2.4 for LaTeX2e as of 16. April 2010
%
\documentclass{llncs}
%
\usepackage{makeidx}  % allows for indexgeneration
\usepackage{llncsdoc}
\usepackage{algorithmic}
\usepackage{amssymb}
\usepackage{graphicx}
\usepackage{wrapfig}
\usepackage{cite}
\usepackage{amsmath}
\usepackage[usenames,dvipsnames]{xcolor}
\usepackage[parfill]{parskip} 
\usepackage{float}
\usepackage{caption}
%\restylefloat{table}

\newcommand{\Mod}[1]{\ (\text{mod}\ #1)}
\newcommand{\todo}[1]{{\color{red}{TODO #1}}}
\captionsetup[table]{name=Tabel}
%\restylefloat{table}
%
\begin{document}
\pagestyle{empty}
%\centerline{\bf UNGIVERSITY OF BUCHAREST}

%\centerline{\bf FACULTY OF MATHEMATICS AND COMPUTER SCIENCE}

%
%
\title{Data Structures and Algorithms}
%
\titlerunning{Data Structures and Algorithms}  % abbreviated title (for running head)
%                                     also used for the TOC unless
%                                     \toctitle is used
%
\author{Drago\c{s} Alin Rotaru}
%TODO - nume final
\authorrunning{and D.A.Rotaru} % abbreviated author list (for running head)
%
%%%% list of authors for the TOC (use if author list has to be modified)
%\tocauthor{}
%
\institute{Computer Science, University of Bucharest, Romania\\
}


\maketitle              % typeset the title of the contribution


\newpage
%----------------------------------------------------------------
%----------------------------------------------------------------
%----------------------------------------------------------------
%----------------------------------------------------------------
%----------------------------------------------------------------

\section{Introduction}
\label{sec:intro}

These seminar notes contain my overview of the Data Structures and Algorithms course held at University of Bucharest. Because the course is based on heavy theoretic lectures, I tried a more practical approach to present some of the notions by discussing problems which arise natural from the main course.

Most of the problems come from a romanian website specialized on programming contests as well as codeforces or topcoder \cite{website:infoarena, website:topcoder, website:codeforces}. Of course, there are more interesting problems to tackle, but unfortunately I limit to the course material although sometimes I will talk about some ad-hoc problems.


\section{Seminar I}
\label{sec:seminar1}

Synthesise first 2 courses:
\begin{itemize}
	\item Basic notions of time and memory complexity.
	\item Stacks and Queues.
\end{itemize}

\subsection{Sketch}
	What is an algorithm? How can we measure the time complexity of a program? Examples (Choosing every pair of elements and erathosthene sieve).
	Introduction to stacks and queues. Details about their implementation and a short tutorial in STL. Can also talk about circular queues and double ended queues.

\subsection{Partial Sums without subtracting}

\subsection{Checking if an expression is has brackets in right order}

\subsection{Emulate a queue using 2 stacks}

\subsection{Editor \cite{website:infoarena/editor}}

\subsection{Alee \cite{website:infoarena/alee}}

\subsection{Trompeta \cite{website:infoarena/trompeta}}

\subsection{Tsunami \cite{website:infoarena/tsunami}}

\subsection{Take-Out \cite{website:edu-pl:take-out}}

\subsection{Devices}
	You are given a row of n devices, each consuming some subset of k<=8 different resources when turned on, and producing some amount of energy when turned on. For each l from 1 to n you need to find the smallest r such that it's possible to turn on some devices from the segment [l;r] such that no two devices turned on consume the same resource, and that the total energy of the devices turned on is at least z \cite{website:petr1}.

\section{Seminar II}
\label{sec:seminar2}

\begin{itemize}
	\item Divide and conquer, merge-sort, estimating complexity
	\item Binary search, fast exponentation and matrix multiplication
\end{itemize}
%
% ---- Bibliography ----
%
%\begin{thebibliography}{5}
%
\clearpage
\bibliographystyle{splncs}
\bibliography{llncs}
%\end{thebibliography}

\end{document}